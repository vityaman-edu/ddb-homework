\documentclass{article}

\usepackage[utf8]{inputenc}
\usepackage[english, russian]{babel}
\usepackage[a4paper, margin=1in]{geometry}
\usepackage{graphicx}
\usepackage{amsmath}
\usepackage{wrapfig}
\usepackage{multirow}
\usepackage{mathtools}
\usepackage{pgfplots}
\usepackage{pgfplotstable}
\usepackage{setspace}
\usepackage{changepage}
\usepackage{caption}
\usepackage{csquotes}
\usepackage{hyperref}
\usepackage{listings}

\pgfplotsset{compat=1.18}
\hypersetup{
  colorlinks = true,
  linkcolor  = blue,
  filecolor  = magenta,      
  urlcolor   = darkgray,
  pdftitle   = ddb-report-2-smirnov-victor-p33131,
}

\definecolor{codegreen}{rgb}{0,0.6,0}
\definecolor{codegray}{rgb}{0.5,0.5,0.5}
\definecolor{codepurple}{rgb}{0.58,0,0.82}
\definecolor{backcolour}{rgb}{0.99,0.99,0.99}

\lstdefinestyle{codestyle}{
  backgroundcolor   = \color{backcolour},   
  commentstyle      = \color{codegreen},
  keywordstyle      = \color{magenta},
  numberstyle       = \tiny\color{codegray},
  stringstyle       = \color{codepurple},
  basicstyle        = \ttfamily\footnotesize,
  breakatwhitespace = false,         
  breaklines        = true,                 
  captionpos        = b,                    
  keepspaces        = true,                 
  numbers           = left,                    
  numbersep         = 5pt,                  
  showspaces        = false,                
  showstringspaces  = false,
  showtabs          = false,                  
  tabsize           = 2
}

\lstset{style=codestyle}

\begin{document}

\begin{titlepage}
    \begin{center}
        \begin{spacing}{1.4}
            \large{Университет ИТМО} \\
            \large{Факультет программной инженерии и компьютерной техники} \\
        \end{spacing}
        \vfill
        \textbf{
            \huge{Распределённые системы хранения данных.} \\
            \huge{Лабораторная работа №2.} \\
        }
    \end{center}
    \vfill
    \begin{center}
        \begin{tabular}{r l}
            Группа:        & P33131                      \\
            Студент:       & Смирнов Виктор Игоревич     \\
            Преподаватель: & Афанасьев Дмитрий Борисович \\
            Вариант:       & 666                         \\
        \end{tabular}
    \end{center}
    \vfill
    \begin{center}
        \begin{large}
            2024
        \end{large}
    \end{center}
\end{titlepage}

\section*{Ключевые слова}

База данных, конфигурация PostgreSQL.

\tableofcontents

\section{Цель работы}

На выделенном узле создать и сконфигурировать новый кластер БД Postgres, саму БД, табличные пространства и новую роль, а также произвести наполнение базы в соответствии с заданием. Отчёт по работе должен содержать все команды по настройке, скрипты, а также измененные строки конфигурационных файлов.

\section{Текст задания}

\subsection{Инициализация кластера БД}

\begin{enumerate}
    \item Директория кластера: \$HOME/kop67
    \item Кодировка: ANSI1251
    \item Локаль: русская
    \item Параметры инициализации задать через аргументы команды
\end{enumerate}

\subsection{Конфигурация и запуск сервера БД}

\begin{enumerate}
    \item Способ подключения: сокет TCP/IP, только localhost
    \item Номер порта: 9666
    \item Остальные способы подключений запретить.
    \item Способ аутентификации клиентов: по имени пользователя
    \item Настроить следующие параметры сервера БД:
    \begin{enumerate}
        \item max\_connections
        \item shared\_buffers
        \item temp\_buffers
        \item work\_mem
        \item checkpoint\_timeout
        \item effective\_cache\_size
        \item fsync
        \item commit\_delay
    \end{enumerate}
    \item Параметры должны быть подобраны в соответствии со сценарием OLAP:
    \begin{itemize}
        \item 5 одновременных пользователей
        \item пакетная запись/чтение данных по 128МБ.
    \end{itemize}
    \item Директория WAL файлов: \$PGDATA/pg\_wal
    \item Формат лог-файлов: .log
    \item Уровень сообщений лога: WARNING
    \item Дополнительно логировать: попытки подключения и завершение сессий
\end{enumerate}

\subsection{Дополнительные табличные пространства и наполнение базы}

\begin{enumerate}
    \item Пересоздать шаблон template1 в новом табличном пространстве: \$HOME/yqy90
    \item На основе template1 создать новую базу: lazyorangehair
    \item Создать новую роль, предоставить необходимые права, разрешить подключение к базе.
    \item От имени новой роли (не администратора) произвести наполнение всех созданных баз тестовыми наборами данных. Все табличные пространства должны использоваться по назначению.
    \item Вывести список всех табличных пространств кластера и содержащиеся в них объекты.
\end{enumerate}

\section{Ход работы}

\subsection{Порядок конфигурации кластера}
\lstinputlisting[language=Bash]{../script/pipeline.sh}

\subsection{Подготовка к установке}
\lstinputlisting[language=Bash]{../script/1-initialization.sh}

\subsection{Инициализация датабазы}
\lstinputlisting[language=Bash]{../script/2-create.sh}

\subsection{Установка конфигурационных файлов}
\lstinputlisting[language=Bash]{../script/3-configuration.sh}

\subsection{Host-based authentication}
\lstinputlisting[language=Bash]{../config/pg_hba.conf}

\subsection{Главный конфигурационный файл}
Кроме требуемых параметров попытался подкрутить из интереса. Было бы интересно узнать, насколько адекватные решения я принял.
\lstinputlisting[language=Bash]{../config/postgresql.conf}

\subsection{Запуск датабазы}
\lstinputlisting[language=Bash]{../script/4-start.sh}

\subsection{Создание табличного пространства и новой датабазы}
\lstinputlisting[language=Bash]{../script/5-recreate-template.sh}

\subsection{Заполнение новой датабазы}
\lstinputlisting[language=Bash]{../script/6-fill-tables.sh}

\subsection{Вывод табличных пространств кластера и объектов}
\lstinputlisting[language=Bash]{../script/7-print.sh}
\lstinputlisting[language=SQL]{../script/tablespace.sql}

\section{Вывод}

Данная лабораторная работа помогла мне изучить конфигурацию PostgreSQL.

\begin{thebibliography}{99}
    \bibitem{PSQLSC} 
        PostgreSQL Documentation, System Catalogs: сайт. - 2024. 
        - URL: \url{https://www.postgresql.org/docs/14/index.html} (дата обращения: 06.04.2024) - Текст : электронный.
\end{thebibliography}

\end{document}