\documentclass{article}

\usepackage[utf8]{inputenc}
\usepackage[english, russian]{babel}
\usepackage[a4paper, margin=1in]{geometry}
\usepackage{graphicx}
\usepackage{amsmath}
\usepackage{wrapfig}
\usepackage{multirow}
\usepackage{mathtools}
\usepackage{pgfplots}
\usepackage{pgfplotstable}
\usepackage{setspace}
\usepackage{changepage}
\usepackage{caption}
\usepackage{csquotes}
\usepackage{hyperref}
\usepackage{listings}

\pgfplotsset{compat=1.18}
\hypersetup{
  colorlinks = true,
  linkcolor  = blue,
  filecolor  = magenta,      
  urlcolor   = darkgray,
  pdftitle   = ddb-report-3-smirnov-victor-p33131,
}

\definecolor{codegreen}{rgb}{0,0.6,0}
\definecolor{codegray}{rgb}{0.5,0.5,0.5}
\definecolor{codepurple}{rgb}{0.58,0,0.82}
\definecolor{backcolour}{rgb}{0.99,0.99,0.99}

\lstdefinestyle{codestyle}{
  backgroundcolor   = \color{backcolour},   
  commentstyle      = \color{codegreen},
  keywordstyle      = \color{magenta},
  numberstyle       = \tiny\color{codegray},
  stringstyle       = \color{codepurple},
  basicstyle        = \ttfamily\footnotesize,
  breakatwhitespace = false,         
  breaklines        = true,                 
  captionpos        = b,                    
  keepspaces        = true,                 
  numbers           = left,                    
  numbersep         = 5pt,                  
  showspaces        = false,                
  showstringspaces  = false,
  showtabs          = false,                  
  tabsize           = 2
}

\lstset{style=codestyle}

\begin{document}

\begin{titlepage}
    \begin{center}
        \begin{spacing}{1.4}
            \large{Университет ИТМО} \\
            \large{Факультет программной инженерии и компьютерной техники} \\
        \end{spacing}
        \vfill
        \textbf{
            \huge{Распределённые системы хранения данных.} \\
            \huge{Лабораторная работа №3.} \\
        }
    \end{center}
    \vfill
    \begin{center}
        \begin{tabular}{r l}
            Группа:        & P33131                      \\
            Студент:       & Смирнов Виктор Игоревич     \\
            Преподаватель: & Афанасьев Дмитрий Борисович \\
            Вариант:       & 736                         \\
        \end{tabular}
    \end{center}
    \vfill
    \begin{center}
        \begin{large}
            2024
        \end{large}
    \end{center}
\end{titlepage}

\section*{Ключевые слова}

База данных, конфигурация PostgreSQL.

\tableofcontents

\section{Цель работы и контекст}

Цель работы - настроить процедуру периодического резервного копирования базы данных, сконфигурированной в ходе выполнения лабораторной работы №2, а также разработать и отладить сценарии восстановления в случае сбоев.

Узел из предыдущей лабораторной работы используется в качестве основного. Новый узел используется в качестве резервного. Учётные данные для подключения к новому узлу выдаёт преподаватель. В сценариях восстановления необходимо использовать копию данных, полученную на первом этапе данной лабораторной работы.

\section{Этап 0. Контекст работы}

В предыдущей лабораторной работы была создана база данных. Приведу здесь скрипты для ее инициализации.

\subsection{Переменные окружения}

\lstinputlisting[
    language=bash,
    caption={Переменные окружения}
]{../db/common/0-env.sh}

\subsection{Конфигурация базы данных}

\lstinputlisting[
    language=bash,
    caption={Конфигурационный файл pg\_hba.conf}
]{../db/primary/pg_hba.conf}

\lstinputlisting[
    language=bash,
    caption={Конфигурационный файл postgresql.conf}
]{../db/primary/postgresql.conf}

\subsection{Создание .pgpass}

\lstinputlisting[
    language=bash,
    caption={Файл .pgpass}
]{../db/common/0-pgpass.sh}

\subsection{Инициализация базы данных}

\lstinputlisting[
    language=bash,
    caption={Инициализация базы данных}
]{../db/primary/1-init.sh}

\subsection{Запуск базы данных}

\lstinputlisting[
    language=bash,
    caption={Запуск базы данных}
]{../db/common/2-start.sh}

\subsection{Настройка базы данных}

\lstinputlisting[
    language=bash,
    caption={Настройка базы данных}
]{../db/primary/3-setup.sh}

\section{Этап 1. Резервное копирование}

\subsection{Задача}

\begin{enumerate}
    \item Настроить резервное копирование с основного узла на резервный следующим образом:
          \begin{enumerate}
              \item Первоначальная полная копия + непрерывное архивирование.
              \item Включить для СУБД режим архивирования WAL;
              \item настроить копирование WAL (scp) на резервный узел;
              \item создать первоначальную резервную копию (pg\_basebackup),
              \item скопировать на резервный узел (rsync).
          \end{enumerate}
    \item Подсчитать, каков будет объем резервных копий спустя месяц работы системы, исходя из следующих условий:
          \begin{enumerate}
              \item Средний объем новых данных в БД за сутки: 650МБ.
              \item Средний объем измененных данных за сутки: 950МБ.
          \end{enumerate}
    \item Проанализировать результаты.
\end{enumerate}

\subsection{Подготовка секретов}

Нам необходимо будет отправлять базовую резервную копию, а так WAL файлы на резервный узел, так что сперва следует сгенерировать и распределить ключи шифрования для безопасной передачи данных между узлами.

\lstinputlisting[
    language=bash,
    caption={Генерация ключей}
]{../db/common/0-ssh-keygen.sh}

Далее я авторизовал публичный ключ primary узла на standby. Теперь можно проверить, что передача данных скорее всего будет работать.

\lstinputlisting[
    language=bash,
    caption={Проверка подключения}
]{../db/common/0-ssh-test.sh}

\subsection{Конфигурация primary узла для резервного копирования}

Включаем архивирование, будем отправлять WAL файлы на standby каждые 16 секунд.

\begin{lstlisting}[
    language=bash, 
    caption={Ключевые строчки в конфигурационном файле}
]
## Archiving

archive_mode    = on
archive_timeout = 16s
archive_command = 'ssh -q <STANDBY_HOST> "test ! -e <STANDBY_WAL_DIR>/%f" && scp %p <STANDBY_HOST>:~/<STANDBY_WAL_DIR>'
\end{lstlisting}

\subsection{Создание базовой резервной копии}

\lstinputlisting[
    language=bash,
    caption={Создание базовой резервной копии}
]{../db/primary/4-backup.sh}

\subsection{Подготовка standby}

\lstinputlisting[
    language=bash,
    caption={Подготовка standby}
]{../db/standby/1-prepare.sh}

\subsection{Полная настройка primary узла}

\lstinputlisting[
    language=bash,
    caption={Полная настройка primary узла}
]{../db/primary/9-full.sh}

\subsection{Действия системного администратора по первоначальной настройке системы}

\begin{enumerate}
    \item Получить конфигурационные файлы системы из репозитория \\
    \url{https://github.com/vityaman-edu/ddb-homework/tree/trunk/lab-3}

    \item Доставить директории \texttt{db/common} и \texttt{db/primary} на \texttt{primary} узел, разместив их в домашней директории пользователя, от лица которого будет запущена система
    
    \item Доставить директории \texttt{db/common} и \texttt{db/standby} на \texttt{standby} узел, разместив их в домашней директории пользователя, от лица которого будет запущена система
    
    \item На \texttt{primary} узле сгенерировать ключи для \texttt{primary} узла при помощи скрипта \\ \texttt{common/0-ssh-keygen.sh} и авторизовать публичный ключ на узле standby 
    
    \item Проверить на \texttt{primary} узле возможность ssh соединения с \texttt{standby} узлом при помощи \\ \texttt{common/0-ssh-test.sh}
    
    \item Подготовить \texttt{standby} узел к резервированию, выполнив на нем \texttt{source common/0-env.sh \&\& sh standby/1-prepare.sh}
    
    \item Запустить \texttt{primary} узел, выполнив на нем \\
    \texttt{
        source common/0-env.sh \&\& 
        sh common/0-pgpass.sh \&\& 
        sh primary/9-full.sh}

    \item Создать базовую резервную копию на \texttt{primary} узле и отправить ее на \texttt{standby} узел: \texttt{sh primary/4-backup.sh}
    
    \item Наполнить данными базу данных на \texttt{primary} узле:
    \texttt{sh primary/5-fill.sh}

    \item Убедиться, что базовая резервная копия и WAL файлы доставлены на \texttt{standby} узел

\end{enumerate}

\section{Этап 2. Потеря основного узла}

\subsection{Задача}

Этот сценарий подразумевает полную недоступность основного узла. Необходимо восстановить работу СУБД на РЕЗЕРВНОМ узле, продемонстрировать успешный запуск СУБД и доступность данных.

\subsection{Восстановление СУБД на резервном узле}

Для этого необходимо просто выполнить 
\texttt{source common/0-env.sh \&\& sh standby/2-restore.sh} 
на \texttt{standby} узле, предварительно убедившись, что в 
директории \texttt{primary/backup/base} находятся файлы базовой резервной копии,
а в директории \texttt{primary/backup/wal} есть WAL сегменты.

\lstinputlisting[
    language=bash,
    caption={Восстановление СУБД на резервном узле}
]{../db/standby/2-restore.sh}

\subsection{Действия на primary узле}

\begin{lstlisting}[
    language=bash,
    caption={Действия на primary узле}
]
postgres0@2d09031f584d:~$ history
1  ls
2  source common/0-env.sh 
3  sh common/0-ssh-keygen.sh 
4  sh common/0-ssh-test.sh ddb-standby
5  sh common/0-pgpass.sh 
6  sh primary/9-full.sh 
7  sh primary/4-backup.sh 
8  sh primary/5-fill.sh 
9  history
\end{lstlisting}

\subsection{Действия на standby узле}

\begin{lstlisting}[
    language=bash,
    caption={Действия на standby узле}
]
postgres0@c33884c20d42:~$ history
2  ls
3  source common/0-env.sh
4  vim .ssh/authorized_keys 
5  sh standby/1-prepare.sh 
6  ls
7  ls primary/backup/base/
8  ls primary/backup/wal/
9  ls primary/backup/base
10  ls primary/backup/wal/
11  sh standby/2-restore.sh 
12  sh common/2-start.sh 
13  bg
14  history
\end{lstlisting}

\begin{lstlisting}[
    language=bash,
    caption={Вывод postgres при старте}
]
postgres0@c33884c20d42:~$ sh common/2-start.sh 
2024-05-28 08:56:18.375 GMT [1088] LOG:  starting PostgreSQL 14.12 (Ubuntu 14.12-1.pgdg20.04+1) on x86_64-pc-linux-gnu, compiled by gcc (Ubuntu 9.4.0-1ubuntu1~20.04.2) 9.4.0, 64-bit
2024-05-28 08:56:18.375 GMT [1088] LOG:  listening on IPv4 address "127.0.0.1", port 9666
2024-05-28 08:56:18.378 GMT [1089] LOG:  database system was interrupted; last known up at 2024-05-28 08:50:55 GMT
cp: cannot stat '/home/postgres0/primary/backup/wal/00000002.history': No such file or directory
2024-05-28 08:56:18.381 GMT [1089] LOG:  starting archive recovery
2024-05-28 08:56:18.392 GMT [1089] LOG:  restored log file "000000010000000000000002" from archive
2024-05-28 08:56:18.396 GMT [1089] LOG:  redo starts at 0/2000028
2024-05-28 08:56:18.396 GMT [1089] LOG:  consistent recovery state reached at 0/2000138
2024-05-28 08:56:18.397 GMT [1088] LOG:  database system is ready to accept read-only connections
2024-05-28 08:56:18.406 GMT [1089] LOG:  restored log file "000000010000000000000003" from archive
2024-05-28 08:56:18.417 GMT [1089] LOG:  restored log file "000000010000000000000004" from archive
cp: cannot stat '/home/postgres0/primary/backup/wal/000000010000000000000005': No such file or directory
2024-05-28 08:56:18.420 GMT [1089] LOG:  redo done at 0/4000148 system usage: CPU: user: 0.00 s, system: 0.00 s, elapsed: 0.02 s
2024-05-28 08:56:18.420 GMT [1089] LOG:  last completed transaction was at log time 2024-05-28 08:51:07.961265+00
2024-05-28 08:56:18.430 GMT [1089] LOG:  restored log file "000000010000000000000004" from archive
cp: cannot stat '/home/postgres0/primary/backup/wal/00000002.history': No such file or directory
2024-05-28 08:56:18.435 GMT [1089] LOG:  selected new timeline ID: 2
2024-05-28 08:56:18.441 GMT [1089] LOG:  archive recovery complete
cp: cannot stat '/home/postgres0/primary/backup/wal/00000001.history': No such file or directory
2024-05-28 08:56:18.450 GMT [1088] LOG:  database system is ready to accept connections
\end{lstlisting}

\begin{lstlisting}[
    language=bash,
    caption={Состояние СУБД на standby узле}
]
$ psql -U "$DDB_PG_USER" -h localhost -p $DDB_PG_PORT -d postges 
postgres=# select * from note_prv;
 id |         content          
----+--------------------------
  1 | Note at postgres
  2 | Another note at postgres
(2 rows)

$ psql -U "$DDB_PG_USER" -h localhost -p $DDB_PG_PORT -d lazyorangehair
lazyorangehair=# select * from note_new;
 id |            content             
----+--------------------------------
  1 | Note at lazyorangehair
  2 | Another note at lazyorangehair
(2 rows)
\end{lstlisting}

Как мы видим, в базе данных сохранились не только данные с базовой копии, но и подтянулись изменения из WAL сегментов.

\section{Этап 3. Повреждение файлов БД}

\subsection{Задача}

Этот сценарий подразумевает потерю данных (например, в результате сбоя диска или файловой системы) при сохранении доступности основного узла. Необходимо выполнить полное восстановление данных из резервной копии и перезапустить СУБД на ОСНОВНОМ узле.

Ход работы:

\begin{enumerate}
    \item Симулировать сбой: удалить с диска директорию любой таблицы со всем содержимым.
    \item Проверить работу СУБД, доступность данных, перезапустить СУБД, проанализировать результаты.
    \item Выполнить восстановление данных из резервной копии, учитывая следующее условие: исходное расположение дополнительных табличных пространств недоступно - разместить в другой директории и скорректировать конфигурацию.
    \item Запустить СУБД, проверить работу и доступность данных, проанализировать результаты.
\end{enumerate}

\subsection{Решение}

TODO

\section{Этап 4. Логическое повреждение данных}

\subsection{Задача}

Этот сценарий подразумевает частичную потерю данных (в результате нежелательной или ошибочной операции) при сохранении доступности основного узла. Необходимо выполнить восстановление данных на ОСНОВНОМ узле следующим способом:

Генерация файла на резервном узле с помощью pg\_dump и последующее применение файла на основном узле.

Ход работы:

\begin{enumerate}
    \item В каждую таблицу базы добавить 2-3 новые строки, зафиксировать результат.
    \item Зафиксировать время и симулировать ошибку: в любой таблице с внешними ключами подменить значения ключей на случайные (INSERT, UPDATE)
    \item Продемонстрировать результат.
    \item Выполнить восстановление данных указанным способом.
    \item Продемонстрировать и проанализировать результат.
\end{enumerate}

\subsection{Решение}

TODO

\section{Вывод}

Данная лабораторная работа помогла мне изучить конфигурацию PostgreSQL.

\begin{thebibliography}{99}
    \bibitem{PSQLSC}
    PostgreSQL Documentation: сайт. - 2024.
    - URL: \url{https://www.postgresql.org/docs/14/index.html} (дата обращения: 06.04.2024) - Текст : электронный.
\end{thebibliography}

\end{document}