\documentclass{article}

\usepackage[utf8]{inputenc}
\usepackage[english, russian]{babel}
\usepackage[a4paper, margin=1in]{geometry}
\usepackage{graphicx}
\usepackage{amsmath}
\usepackage{wrapfig}
\usepackage{multirow}
\usepackage{mathtools}
\usepackage{pgfplots}
\usepackage{pgfplotstable}
\usepackage{setspace}
\usepackage{changepage}
\usepackage{caption}
\usepackage{csquotes}
\usepackage{hyperref}
\usepackage{listings}

\pgfplotsset{compat=1.18}
\hypersetup{
  colorlinks = true,
  linkcolor  = blue,
  filecolor  = magenta,      
  urlcolor   = darkgray,
  pdftitle   = ddb-report-3-smirnov-victor-p33131,
}

\definecolor{codegreen}{rgb}{0,0.6,0}
\definecolor{codegray}{rgb}{0.5,0.5,0.5}
\definecolor{codepurple}{rgb}{0.58,0,0.82}
\definecolor{backcolour}{rgb}{0.99,0.99,0.99}

\lstdefinestyle{codestyle}{
  backgroundcolor   = \color{backcolour},   
  commentstyle      = \color{codegreen},
  keywordstyle      = \color{magenta},
  numberstyle       = \tiny\color{codegray},
  stringstyle       = \color{codepurple},
  basicstyle        = \ttfamily\footnotesize,
  breakatwhitespace = false,         
  breaklines        = true,                 
  captionpos        = b,                    
  keepspaces        = true,                 
  numbers           = left,                    
  numbersep         = 5pt,                  
  showspaces        = false,                
  showstringspaces  = false,
  showtabs          = false,                  
  tabsize           = 2
}

\lstset{style=codestyle}

\begin{document}

\begin{titlepage}
    \begin{center}
        \begin{spacing}{1.4}
            \large{Университет ИТМО} \\
            \large{Факультет программной инженерии и компьютерной техники} \\
        \end{spacing}
        \vfill
        \textbf{
            \huge{Распределённые системы хранения данных.} \\
            \huge{Лабораторная работа №3.} \\
        }
    \end{center}
    \vfill
    \begin{center}
        \begin{tabular}{r l}
            Группа:        & P33131                      \\
            Студент:       & Смирнов Виктор Игоревич     \\
            Преподаватель: & Афанасьев Дмитрий Борисович \\
            Вариант:       & 736                         \\
        \end{tabular}
    \end{center}
    \vfill
    \begin{center}
        \begin{large}
            2024
        \end{large}
    \end{center}
\end{titlepage}

\section*{Ключевые слова}

База данных, конфигурация PostgreSQL.

\tableofcontents

\section{Цель работы и контекст}

Цель работы - настроить процедуру периодического резервного копирования базы данных, сконфигурированной в ходе выполнения лабораторной работы №2, а также разработать и отладить сценарии восстановления в случае сбоев.

Узел из предыдущей лабораторной работы используется в качестве основного. Новый узел используется в качестве резервного. Учётные данные для подключения к новому узлу выдаёт преподаватель. В сценариях восстановления необходимо использовать копию данных, полученную на первом этапе данной лабораторной работы.

\section{Этап 1. Резервное копирование}

\subsection{Задача}

\begin{enumerate}
    \item Настроить резервное копирование с основного узла на резервный следующим образом:
    \begin{enumerate}
        \item Первоначальная полная копия + непрерывное архивирование.
        \item Включить для СУБД режим архивирования WAL;
        \item настроить копирование WAL (scp) на резервный узел;
        \item создать первоначальную резервную копию (pg\_basebackup),
        \item скопировать на резервный узел (rsync).
    \end{enumerate}
    \item Подсчитать, каков будет объем резервных копий спустя месяц работы системы, исходя из следующих условий:
    \begin{enumerate}
        \item Средний объем новых данных в БД за сутки: 650МБ.
        \item Средний объем измененных данных за сутки: 950МБ.
    \end{enumerate}
    \item Проанализировать результаты.
\end{enumerate}

\subsection{Решение}

TODO

\section{Этап 2. Потеря основного узла}

\subsection{Задача}

Этот сценарий подразумевает полную недоступность основного узла. Необходимо восстановить работу СУБД на РЕЗЕРВНОМ узле, продемонстрировать успешный запуск СУБД и доступность данных.

\subsection{Решение}

TODO

\section{Этап 3. Повреждение файлов БД}

\subsection{Задача}

Этот сценарий подразумевает потерю данных (например, в результате сбоя диска или файловой системы) при сохранении доступности основного узла. Необходимо выполнить полное восстановление данных из резервной копии и перезапустить СУБД на ОСНОВНОМ узле.


Ход работы:

\begin{enumerate}
    \item Симулировать сбой: удалить с диска директорию любой таблицы со всем содержимым.
    \item Проверить работу СУБД, доступность данных, перезапустить СУБД, проанализировать результаты.
    \item Выполнить восстановление данных из резервной копии, учитывая следующее условие: исходное расположение дополнительных табличных пространств недоступно - разместить в другой директории и скорректировать конфигурацию.
    \item Запустить СУБД, проверить работу и доступность данных, проанализировать результаты.
\end{enumerate}

\subsection{Решение}

TODO

\section{Этап 4. Логическое повреждение данных}

\subsection{Задача}

Этот сценарий подразумевает частичную потерю данных (в результате нежелательной или ошибочной операции) при сохранении доступности основного узла. Необходимо выполнить восстановление данных на ОСНОВНОМ узле следующим способом:

Генерация файла на резервном узле с помощью pg\_dump и последующее применение файла на основном узле.

Ход работы:

\begin{enumerate}
    \item В каждую таблицу базы добавить 2-3 новые строки, зафиксировать результат.
    \item Зафиксировать время и симулировать ошибку: в любой таблице с внешними ключами подменить значения ключей на случайные (INSERT, UPDATE)
    \item Продемонстрировать результат.
    \item Выполнить восстановление данных указанным способом.
    \item Продемонстрировать и проанализировать результат.
\end{enumerate}

\subsection{Решение}

TODO

\section{Вывод}

Данная лабораторная работа помогла мне изучить конфигурацию PostgreSQL.

\begin{thebibliography}{99}
    \bibitem{PSQLSC} 
        PostgreSQL Documentation: сайт. - 2024. 
        - URL: \url{https://www.postgresql.org/docs/14/index.html} (дата обращения: 06.04.2024) - Текст : электронный.
\end{thebibliography}

\end{document}