\documentclass{article}

\usepackage[utf8]{inputenc}
\usepackage[english, russian]{babel}
\usepackage[a4paper, margin=1in]{geometry}
\usepackage{graphicx}
\usepackage{amsmath}
\usepackage{wrapfig}
\usepackage{multirow}
\usepackage{mathtools}
\usepackage{pgfplots}
\usepackage{pgfplotstable}
\usepackage{setspace}
\usepackage{changepage}
\usepackage{caption}
\usepackage{csquotes}
\usepackage{hyperref}
\usepackage{listings}

\pgfplotsset{compat=1.18}
\hypersetup{
  colorlinks = true,
  linkcolor  = blue,
  filecolor  = magenta,      
  urlcolor   = darkgray,
  pdftitle   = ddb-report-3-smirnov-victor-p33131,
}

\definecolor{codegreen}{rgb}{0,0.6,0}
\definecolor{codegray}{rgb}{0.5,0.5,0.5}
\definecolor{codepurple}{rgb}{0.58,0,0.82}
\definecolor{backcolour}{rgb}{0.99,0.99,0.99}

\lstdefinestyle{codestyle}{
  backgroundcolor   = \color{backcolour},   
  commentstyle      = \color{codegreen},
  keywordstyle      = \color{magenta},
  numberstyle       = \tiny\color{codegray},
  stringstyle       = \color{codepurple},
  basicstyle        = \ttfamily\footnotesize,
  breakatwhitespace = false,         
  breaklines        = true,                 
  captionpos        = b,                    
  keepspaces        = true,                 
  numbers           = left,                    
  numbersep         = 5pt,                  
  showspaces        = false,                
  showstringspaces  = false,
  showtabs          = false,                  
  tabsize           = 2
}

\lstset{style=codestyle}

\begin{document}

\begin{titlepage}
    \begin{center}
        \begin{spacing}{1.4}
            \large{Университет ИТМО} \\
            \large{Факультет программной инженерии и компьютерной техники} \\
        \end{spacing}
        \vfill
        \textbf{
            \huge{Распределённые системы хранения данных.} \\
            \huge{Лабораторная работа №3.} \\
        }
    \end{center}
    \vfill
    \begin{center}
        \begin{tabular}{r l}
            Группа:        & P33131                      \\
            Студент:       & Смирнов Виктор Игоревич     \\
            Преподаватель: & Афанасьев Дмитрий Борисович \\
            Вариант:       & 736                         \\
        \end{tabular}
    \end{center}
    \vfill
    \begin{center}
        \begin{large}
            2024
        \end{large}
    \end{center}
\end{titlepage}

\section*{Ключевые слова}

База данных, конфигурация PostgreSQL.

\tableofcontents

\section{Цель работы и контекст}

Цель работы - настроить процедуру периодического резервного копирования базы данных, сконфигурированной в ходе выполнения лабораторной работы №2, а также разработать и отладить сценарии восстановления в случае сбоев.

Узел из предыдущей лабораторной работы используется в качестве основного. Новый узел используется в качестве резервного. Учётные данные для подключения к новому узлу выдаёт преподаватель. В сценариях восстановления необходимо использовать копию данных, полученную на первом этапе данной лабораторной работы.

\section{Этап 0. Контекст работы}

В предыдущей лабораторной работы была создана база данных. Приведу здесь скрипты для ее инициализации.

\subsection{Переменные окружения}

\lstinputlisting[
    language=bash,
    caption={Переменные окружения}
]{../db/common/env.sh}

\subsection{Конфигурация базы данных}

\lstinputlisting[
    language=bash,
    caption={Конфигурационный файл pg\_hba.conf}
]{../db/primary/pg_hba.conf}

\lstinputlisting[
    language=bash,
    caption={Конфигурационный файл postgresql.conf}
]{../db/primary/postgresql.conf}

\subsection{Создание .pgpass}

\lstinputlisting[
    language=bash,
    caption={Файл .pgpass}
]{../db/common/pgpass.sh}

\subsection{Инициализация базы данных}

\lstinputlisting[
    language=bash,
    caption={Инициализация базы данных}
]{../db/primary/init.sh}

\subsection{Запуск базы данных}

\lstinputlisting[
    language=bash,
    caption={Запуск базы данных}
]{../db/common/start.sh}

\subsection{Настройка базы данных}

\lstinputlisting[
    language=bash,
    caption={Настройка базы данных}
]{../db/primary/setup.sh}

\section{Этап 1. Резервное копирование}

\subsection{Задача}

\begin{enumerate}
    \item Настроить резервное копирование с основного узла на резервный следующим образом:
          \begin{enumerate}
              \item Первоначальная полная копия + непрерывное архивирование.
              \item Включить для СУБД режим архивирования WAL;
              \item настроить копирование WAL (scp) на резервный узел;
              \item создать первоначальную резервную копию (pg\_basebackup),
              \item скопировать на резервный узел (rsync).
          \end{enumerate}
    \item Подсчитать, каков будет объем резервных копий спустя месяц работы системы, исходя из следующих условий:
          \begin{enumerate}
              \item Средний объем новых данных в БД за сутки: 650МБ.
              \item Средний объем измененных данных за сутки: 950МБ.
          \end{enumerate}
    \item Проанализировать результаты.
\end{enumerate}

\subsection{Подготовка секретов}

Нам необходимо будет отправлять базовую резервную копию, а так WAL файлы на резервный узел, так что сперва следует сгенерировать и распределить ключи шифрования для безопасной передачи данных между узлами.

\lstinputlisting[
    language=bash,
    caption={Генерация ключей}
]{../db/common/ssh-keygen.sh}

Далее я авторизовал публичный ключ primary узла на standby. Теперь можно проверить, что передача данных скорее всего будет работать.

\lstinputlisting[
    language=bash,
    caption={Проверка подключения}
]{../db/common/ssh-test.sh}

\subsection{Конфигурация primary узла для резервного копирования}

Включаем архивирование, будем отправлять WAL файлы на standby каждые 16 секунд.

\begin{lstlisting}[
    language=bash, 
    caption={Ключевые строчки в конфигурационном файле}
]
## Archiving

archive_mode    = on
archive_timeout = 16s
archive_command = 'ssh -q <STANDBY_HOST> "test ! -e <STANDBY_WAL_DIR>/%f" && scp %p <STANDBY_HOST>:~/<STANDBY_WAL_DIR>'
\end{lstlisting}

\subsection{Создание базовой резервной копии}

\lstinputlisting[
    language=bash,
    caption={Создание базовой резервной копии}
]{../db/primary/backup.sh}

\subsection{Подготовка standby}

\lstinputlisting[
    language=bash,
    caption={Подготовка standby}
]{../db/standby/prepare.sh}

\subsection{Полная настройка primary узла}

\lstinputlisting[
    language=bash,
    caption={Полная настройка primary узла}
]{../db/primary/full.sh}

\subsection{Действия системного администратора по первоначальной настройке системы}

\begin{enumerate}
    \item Получить конфигурационные файлы системы из репозитория \\
          \url{https://github.com/vityaman-edu/ddb-homework/tree/trunk/lab-3}

    \item Доставить директории \texttt{db/common} и \texttt{db/primary} на \texttt{primary} узел, разместив их в домашней директории пользователя, от лица которого будет запущена система

    \item Доставить директории \texttt{db/common} и \texttt{db/standby} на \texttt{standby} узел, разместив их в домашней директории пользователя, от лица которого будет запущена система

    \item На \texttt{primary} узле сгенерировать ключи для \texttt{primary} узла при помощи скрипта \\ \texttt{common/ssh-keygen.sh} и авторизовать публичный ключ на узле standby

    \item Проверить на \texttt{primary} узле возможность ssh соединения с \texttt{standby} узлом при помощи \\ \texttt{common/ssh-test.sh}

    \item Подготовить \texttt{standby} узел к резервированию, выполнив на нем \texttt{source common/env.sh \&\& sh standby/prepare.sh}

    \item Запустить \texttt{primary} узел, выполнив на нем \\
          \texttt{
              source common/env.sh \&\&
              sh common/pgpass.sh \&\&
              sh primary/full.sh}

    \item Создать базовую резервную копию на \texttt{primary} узле и отправить ее на \texttt{standby} узел: \texttt{sh primary/backup.sh}

    \item Наполнить данными базу данных на \texttt{primary} узле:
          \texttt{sh primary/fill.sh}

    \item Убедиться, что базовая резервная копия и WAL файлы доставлены на \texttt{standby} узел

\end{enumerate}

\subsection{Объем резервных копий спустя месяц работы}

\begin{enumerate}
    \item Средний объем новых данных в БД за сутки: N = 650МБ
    \item Средний объем измененных данных за сутки: M = 950МБ
\end{enumerate}

Базовая резервная копия у меня весит $B = 50 \; \texttt{MB}$. 

Сперва оценим величину грубо:

$$B + (N + M) \cdot 30 = 46 \; \texttt{GB} $$

Оценка является скорее всего не самой достоверной, ведь возможно использование алгоритмов сжатия для понижения объемов хранимых данных. Не берусь оценивать ислледуемую величину, ведь это лучше доказывать экспериментально.

\section{Этап 2. Потеря основного узла}

\subsection{Задача}

Этот сценарий подразумевает полную недоступность основного узла. Необходимо восстановить работу СУБД на РЕЗЕРВНОМ узле, продемонстрировать успешный запуск СУБД и доступность данных.

\subsection{Восстановление СУБД на резервном узле}

Для этого необходимо просто выполнить
\texttt{source common/env.sh \&\& sh common/restore.sh standby anon-tblspc}
на \texttt{standby} узле, предварительно убедившись, что в
директории \texttt{primary/backup/base} находятся файлы базовой резервной копии,
а в директории \texttt{primary/backup/wal} есть WAL сегменты.

\lstinputlisting[
    language=bash,
    caption={Восстановление СУБД}
]{../db/common/restore.sh}

\subsection{Действия на primary узле}

\begin{lstlisting}[
    language=bash,
    caption={Действия на primary узле}
]
postgres0@2d09031f584d:~$ history
1  ls
2  source common/env.sh 
3  sh common/ssh-keygen.sh 
4  sh common/ssh-test.sh ddb-standby
5  sh common/pgpass.sh 
6  sh primary/full.sh 
7  sh primary/backup.sh 
8  sh primary/fill.sh 
9  history
\end{lstlisting}

\subsection{Действия на standby узле}

\begin{lstlisting}[
    language=bash,
    caption={Действия на standby узле}
]
postgres0@c33884c20d42:~$ history
2  ls
3  source common/env.sh
4  vim .ssh/authorized_keys 
5  sh standby/prepare.sh 
6  ls
7  ls primary/backup/base/
8  ls primary/backup/wal/
9  ls primary/backup/base
10  ls primary/backup/wal/
11  sh standby/restore.sh 
12  sh common/start.sh 
13  bg
14  history
\end{lstlisting}

\texttt{
    \$ sh common/start.sh \\
    2024-05-28 17:29:22.989 GMT [61131] СООБЩЕНИЕ:  запускается PostgreSQL 14.2 on amd64-portbld-freebsd13.0, compiled by FreeBSD clang version 11.0.1 (git@github.com:llvm/llvm-project.git llvmorg-11.0.1-0-g43ff75f2c3fe), 64-bit\\
    2024-05-28 17:29:22.989 GMT [61131] СООБЩЕНИЕ:  для приёма подключений по адресу IPv4 "127.0.0.1" открыт порт 9666\\
    2024-05-28 17:29:22.991 GMT [61132] СООБЩЕНИЕ:  работа системы БД была прервана; последний момент работы: 2024-05-28 17:19:51 GMT\\
    cp: /var/db/postgres1/primary/backup/wal/00000002.history: No such file or directory\\
    2024-05-28 17:29:23.012 GMT [61132] СООБЩЕНИЕ:  начинается восстановление архива\\
    2024-05-28 17:29:23.014 GMT [61132] СООБЩЕНИЕ:  файл журнала "000000010000000000000002" восстановлен из архива\\
    2024-05-28 17:29:23.017 GMT [61132] СООБЩЕНИЕ:  запись REDO начинается со смещения 0/2000028\\
    2024-05-28 17:29:23.017 GMT [61132] СООБЩЕНИЕ:  согласованное состояние восстановления достигнуто по смещению 0/2000138\\
    2024-05-28 17:29:23.018 GMT [61131] СООБЩЕНИЕ:  система БД готова принимать подключения в режиме "только чтение"\\
    2024-05-28 17:29:23.019 GMT [61132] СООБЩЕНИЕ:  файл журнала "000000010000000000000003" восстановлен из архива\\
    2024-05-28 17:29:23.029 GMT [61132] СООБЩЕНИЕ:  файл журнала "000000010000000000000004" восстановлен из архива\\
    cp: /var/db/postgres1/primary/backup/wal/000000010000000000000005: No such file or directory\\
    2024-05-28 17:29:23.031 GMT [61132] СООБЩЕНИЕ:  записи REDO обработаны до смещения 0/4000110, нагрузка системы: CPU: пользов.: 0.00 с, система: 0.00 с, прошло: 0.01 с\\
    2024-05-28 17:29:23.031 GMT [61132] СООБЩЕНИЕ:  последняя завершённая транзакция была выполнена в 2024-05-28 17:20:13.468706+00\\
    2024-05-28 17:29:23.034 GMT [61132] СООБЩЕНИЕ:  файл журнала "000000010000000000000004" восстановлен из архива\\
    cp: /var/db/postgres1/primary/backup/wal/00000002.history: No such file or directory\\
    2024-05-28 17:29:23.035 GMT [61132] СООБЩЕНИЕ:  выбранный ID новой линии времени: 2\\
    2024-05-28 17:29:23.043 GMT [61132] СООБЩЕНИЕ:  восстановление архива завершено\\
    cp: /var/db/postgres1/primary/backup/wal/00000001.history: No such file or directory\\
    2024-05-28 17:29:23.054 GMT [61131] СООБЩЕНИЕ:  система БД готова принимать подключения\\
}

\begin{lstlisting}[
    language=bash,
    caption={Состояние СУБД на standby узле}
]
$ sh common/connect.sh postges 
postgres=# select * from note_prv;
 id |         content          
----+--------------------------
  1 | Note at postgres
  2 | Another note at postgres
(2 rows)

$ sh common/connect.sh lazyorangehair
lazyorangehair=# select * from note_new;
 id |            content             
----+--------------------------------
  1 | Note at lazyorangehair
  2 | Another note at lazyorangehair
(2 rows)
\end{lstlisting}

Как мы видим, в базе данных сохранились не только данные с базовой копии, но и подтянулись изменения из WAL сегментов.

\section{Этап 3. Повреждение файлов БД}

\subsection{Задача}

Этот сценарий подразумевает потерю данных (например, в результате сбоя диска или файловой системы) при сохранении доступности основного узла. Необходимо выполнить полное восстановление данных из резервной копии и перезапустить СУБД на ОСНОВНОМ узле.

Ход работы:

\begin{enumerate}
    \item Симулировать сбой: удалить с диска директорию любой таблицы со всем содержимым.
    \item Проверить работу СУБД, доступность данных, перезапустить СУБД, проанализировать результаты.
    \item Выполнить восстановление данных из резервной копии, учитывая следующее условие: исходное расположение дополнительных табличных пространств недоступно - разместить в другой директории и скорректировать конфигурацию.
    \item Запустить СУБД, проверить работу и доступность данных, проанализировать результаты.
\end{enumerate}

\subsection{Решение}

Попробуем удалить таблицу \texttt{note\_prv}. Для этого узнаем, где представлены ее данные в файловой системе.

\begin{lstlisting}[
    language=bash,
    caption={Получаем физическую локацию таблицы}
]
$ psql -h localhost -p "$DDB_PG_PORT" "$DDB_PG_DATABASE"
postgres=# SELECT pg_relation_filepath('note_prv');
 pg_relation_filepath 
----------------------
 base/14115/16389
(1 stroka)
\end{lstlisting}

Начинаем уничтожение.

\begin{lstlisting}[
    language=bash,
    caption={Удаляем файл таблицы}
]
$ rm $PGDATA/base/14374/16389

\end{lstlisting}

Наблюдаем эффекты.

\texttt{
    \$ psql -h localhost -p "\$DDB\_PG\_PORT" "\$DDB\_PG\_DATABASE" \\
    postgres=\# select * from note\_prv; \\
    ERROR:  could not open file "base/14374/16389": No such file or directory \\
}

Попробуем перезапустить базу данных.

\begin{lstlisting}[
    language=bash,
    caption={Перезапускаем датабазу}
]
$ sh common/stop.sh
waiting for server to shut down.... done
server stopped

$ sh common/start.sh
2024-05-28 10:52:21.905 GMT [9200] LOG:  starting PostgreSQL 14.12 (Ubuntu 14.12-1.pgdg20.04+1) on x86_64-pc-linux-gnu, compiled by gcc (Ubuntu 9.4.0-1ubuntu1~20.04.2) 9.4.0, 64-bit
2024-05-28 10:52:21.905 GMT [9200] LOG:  listening on IPv4 address "127.0.0.1", port 9666
2024-05-28 10:52:21.908 GMT [9206] LOG:  database system was shut down at 2024-05-28 10:52:16 GMT
2024-05-28 10:52:21.916 GMT [9200] LOG:  database system is ready to accept connections

$ psql -h localhost -p "$DDB_PG_PORT" "$DDB_PG_DATABASE"
postgres=# select * from note_prv;
2024-05-28 10:54:18.117 GMT [9230] ERROR:  could not open file "base/14374/16389": No such file or directory
2024-05-28 10:54:18.117 GMT [9230] STATEMENT:  select * from note_prv;
ERROR:  could not open file "base/14374/16389": No such file or directory
\end{lstlisting}

Видим, что postgres не заметил пропажи, значит не проверил целостность файлов. Наверное, должен существовать способ осуществить проверку целостность его файлов. \texttt{pg\_checksums} не обнаружил проблему. По идее, можно использовать для проверки утилиту \texttt{pg\_verifybackup}, которая проверит, соответствует ли содержимое \texttt{PGDATA} заданному бэкапу.

Скачиваем бэкап с \texttt{standby}, сносим нашу больную датабазу, запускаем восстановление и проверяем результат.

\begin{lstlisting}[
    language=bash,
    caption={Восстанавливаем базу данных}
]
$ sh primary/download.sh 

$ ls primary/backup/base primary/backup/wal/
primary/backup/base:
16384.tar  backup_manifest  base.tar

primary/backup/wal/:
000000010000000000000001
000000010000000000000002
000000010000000000000002.00000028.backup
000000010000000000000003
000000010000000000000004
000000010000000000000005

$ sh common/restore.sh 

\end{lstlisting}

\texttt{
\$ sh common/start.sh  \\
2024-05-28 17:42:12.835 GMT [61895] СООБЩЕНИЕ:  запускается PostgreSQL 14.2 on amd64-portbld-freebsd13.0, compiled by FreeBSD clang version 11.0.1 (git@github.com:llvm/llvm-project.git llvmorg-11.0.1-0-g43ff75f2c3fe), 64-bit\\
2024-05-28 17:42:12.835 GMT [61895] СООБЩЕНИЕ:  для приёма подключений по адресу IPv4 "127.0.0.1" открыт порт 9666\\
2024-05-28 17:42:12.838 GMT [61896] СООБЩЕНИЕ:  работа системы БД была прервана; последний момент работы: 2024-05-28 17:19:51 GMT\\
cp: /var/db/postgres0/primary/backup/wal/00000002.history: No such file or directory\\
2024-05-28 17:42:12.840 GMT [61896] СООБЩЕНИЕ:  начинается восстановление архива\\
2024-05-28 17:42:12.842 GMT [61896] СООБЩЕНИЕ:  файл журнала "000000010000000000000002" восстановлен из архива\\
2024-05-28 17:42:12.843 GMT [61896] СООБЩЕНИЕ:  запись REDO начинается со смещения 0/2000028\\
2024-05-28 17:42:12.843 GMT [61896] СООБЩЕНИЕ:  согласованное состояние восстановления достигнуто по смещению 0/2000138\\
2024-05-28 17:42:12.844 GMT [61895] СООБЩЕНИЕ:  система БД готова принимать подключения в режиме "только чтение"\\
2024-05-28 17:42:12.845 GMT [61896] СООБЩЕНИЕ:  файл журнала "000000010000000000000003" восстановлен из архива\\
2024-05-28 17:42:12.852 GMT [61896] СООБЩЕНИЕ:  файл журнала "000000010000000000000004" восстановлен из архива\\
2024-05-28 17:42:12.854 GMT [61896] СООБЩЕНИЕ:  файл журнала "000000010000000000000005" восстановлен из архива\\
2024-05-28 17:42:12.856 GMT [61896] СООБЩЕНИЕ:  файл журнала "000000010000000000000006" восстановлен из архива\\
cp: /var/db/postgres0/primary/backup/wal/000000010000000000000007: No such file or directory\\
2024-05-28 17:42:12.858 GMT [61896] СООБЩЕНИЕ:  записи REDO обработаны до смещения 0/60000D8, нагрузка системы: CPU: пользов.: 0.00 с, система: 0.00 с, прошло: 0.01 с\\
2024-05-28 17:42:12.858 GMT [61896] СООБЩЕНИЕ:  последняя завершённая транзакция была выполнена в 2024-05-28 17:20:13.468706+00\\
2024-05-28 17:42:12.860 GMT [61896] СООБЩЕНИЕ:  файл журнала "000000010000000000000006" восстановлен из архива\\
cp: /var/db/postgres0/primary/backup/wal/00000002.history: No such file or directory\\
2024-05-28 17:42:12.862 GMT [61896] СООБЩЕНИЕ:  выбранный ID новой линии времени: 2\\
2024-05-28 17:42:12.865 GMT [61896] СООБЩЕНИЕ:  восстановление архива завершено\\
cp: /var/db/postgres0/primary/backup/wal/00000001.history: No such file or directory\\
2024-05-28 17:42:12.876 GMT [61895] СООБЩЕНИЕ:  система БД готова принимать подключения\\
\$ psql -h localhost -p "\$DDB\_PG\_PORT" "\$DDB\_PG\_DATABASE"\\
postgres=\# select * from note\_prv;\\
 id |         content          \\
----+--------------------------\\
  1 | Note at postgres\\
  2 | Another note at postgres\\
(2 rows)
}

Починили.

Вторая часть выполняется аналогично, но нужно вызывать восстановление через \texttt{sh common/restore.bash primary anon-tblspc}.

\section{Этап 4. Логическое повреждение данных}

\subsection{Задача}

Этот сценарий подразумевает частичную потерю данных (в результате нежелательной или ошибочной операции) при сохранении доступности основного узла. Необходимо выполнить восстановление данных на ОСНОВНОМ узле следующим способом:

Генерация файла на резервном узле с помощью pg\_dump и последующее применение файла на основном узле.

Ход работы:

\begin{enumerate}
    \item В каждую таблицу базы добавить 2-3 новые строки, зафиксировать результат.
    \item Зафиксировать время и симулировать ошибку: в любой таблице с внешними ключами подменить значения ключей на случайные (INSERT, UPDATE)
    \item Продемонстрировать результат.
    \item Выполнить восстановление данных указанным способом.
    \item Продемонстрировать и проанализировать результат.
\end{enumerate}

\subsection{Фиксируем состояние БД}

Сперва запустим вставку нескольких строк на основном узле через скрипт, после чего фиксируем текущее время 2024-05-28 17:46:57.818.

\lstinputlisting[
    language=bash,
    caption={Вставка дополнительных строк}
]{../db/primary/fill-extra.sh}

Теперь испортим. Не очень понятно, как я должен портить данные, поэтому просто вставлю строку-индикатор: \texttt{insert into note\_prv (content) values ('aaaaaaaaaaa');
}

\subsection{Восстановление дампа из бэкапа на резервном узле}

Теперь летим на резервный узел и там восстанавливаем бд до заданного момента времени: \texttt{sh standby/backup2dump.sh "2024-05-28 17:46:57"}.

\lstinputlisting[
    language=bash,
    caption={Скрипт backup2dump}
]{../db/standby/backup2dump.sh}

\lstinputlisting[
    language=bash,
    caption={Скрипт sql-dump}
]{../db/common/sql-dump.sh}

После этого в директории с бэкапами появились нужные нам архивы. 

Теперь летим на главный узел и запускаем наш кошмар там: \texttt{sh primary/donwload.sh \&\& sh common/sql-restore.sh}.

\lstinputlisting[
    language=bash,
    caption={Скрипт sql-restore}
]{../db/common/sql-restore.sh}

Проверяем, что в таблицах.

\texttt{
postgres=\# select * from note\_prv;\\
 id |         content          \\
  1 | Note at postgres\\
  2 | Another note at postgres\\
 35 | The first testing note\\
 36 | The second testing note\\
 37 | The third testing note
}

\section{Вывод}

Данная лабораторная работа помогла мне по-настоящему ощутить, какого это - быть администратором PostgreSQL. Я научился создавать резервные копии и накатывать их. В предложенной реализации есть проблема, связанная с тем, что я не накатываю валы переодически, из-за чего через 20 лет я не смогу восстановить БД за разумное время. Также я научился писать переносимые шел скрипты для автоматизации развертывания СУБД.

\section{Бонус}

\lstinputlisting{../Dockerfile}

\begin{thebibliography}{99}
    \bibitem{PSQLSC}
    PostgreSQL Documentation: сайт. - 2024.
    - URL: \url{https://www.postgresql.org/docs/14/index.html} (дата обращения: 06.04.2024) - Текст : электронный.
\end{thebibliography}

\end{document}